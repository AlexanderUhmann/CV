\documentclass[grey]{hipstercv}
% available options are: darkhipster, lighthipster, pastel, allblack, grey, verylight
\usepackage[utf8]{inputenc}
\usepackage[default]{raleway}
\usepackage[margin=1cm, a4paper]{geometry}

\usepackage{hyperref}
\hypersetup{
	colorlinks = true,
	%linkcolor = green,
	%filecolor = magenta,      
	%urlcolor = cyan,
	urlcolor = red,
	%pdftitle = {Overleaf Example},
	%pdfpagemode=FullScreen,
}


%------------------------------------------------------------------ Variablen

\newlength{\rightcolwidth}
\newlength{\leftcolwidth}
\setlength{\leftcolwidth}{0.3\textwidth}
\setlength{\rightcolwidth}{0.65\textwidth}

%------------------------------------------------------------------
\title{Hipster-CV}
\author{\LaTeX{} Ninja}
\date{March 2019}

\pagestyle{empty}
\begin{document}


\thispagestyle{empty}
%-------------------------------------------------------------

\section*{Start}



\header{\bg{headerfontbox}{headerfontboxfont}{Mathematiker} }{\bgupper{headerfontbox}{headerfontboxfont}{\bfseries\Huge Alexander Uhmann}}{\bg{headerfontbox}{headerfontboxfont}{\large Data Scientist}}{Alex_2023.png}{headerblue}{3.5cm}{2cm}


\begin{flushright}
	{\color{white}\Large 
		%\icon{\faEnvelopeO}{cvpurple}{} Sachsen 
		%\icon{\faMapMarker}{cvpurple}{} 04207 Leipzig
	    \icon{\faPhone}{cvpurple}{} 0172/1854194\\ %\newline\icon{\faAt}{cvpurple}{}
		\vspace{0.1em}
	    \large\icon{\faEnvelopeO}{cvpurple}{}alexander.uhmann@gmail.com
	    
    }			
\end{flushright}




%------------------------------------------------

% hier muss die "unsichtbare" Überschrift rein, weil er sonst nicht die Paracols startet... komisch...
\subsection*{}
\vspace{0.1em}

\setlength{\columnsep}{0.cm}
\columnratio{0.33}[0.55]
\begin{paracol}{2}
\hbadness5000
\backgroundcolor{c[1]}[rgb]{1,1,0.8} % cream yellow for column-1 
%\backgroundcolor{g}[rgb]{0.8,1,1} % \backgroundcolor{l}[rgb]{0,0,0.7} % dark blue for left margin

\paracolbackgroundoptions

% 0.9,0.9,0.9 -- 0.8,0.8,0.8


\footnotesize
{\setasidefontcolour
\bgupper{cvgreen}{white}{Facts} \\

{\Large

%\bg{cvgreen}{white}{personal} \\

\begin{tabular}{ll}
%\faMale&Alexander Uhmann \\
\faGlobe&Nationalität: Deutsch  \\
\faBirthdayCake&07/01/1983 \\
\faMapMarker&04207 Leipzig  \\
\end{tabular}
}
\bigskip

%\bg{cvgreen}{white}{Areas of specialization} \\
\bg{cvgreen}{white}{{\huge Spezialisierung}} \\

{\Large
Machine learning\\ 
Deep learning
}

\bigskip

\bgupper{cvgreen}{white}{Skills} \\
{\large
\bg{cvgreen}{white}{Languages}\\
\bigskip
\begin{minipage}[t]{\leftcolwidth}
\begin{tabular}{l | ll}
\textbf{Russisch} &  & {\phantom{x}\footnotesize Muttersprache} \\
\textbf{Deutsch} & B2-C1 & \pictofraction{\faCircle}{cvpurple}{3}{black!30}{1}{\tiny} \\
\textbf{Englisch} & B1-B2 & \pictofraction{\faCircle}{cvpurple}{2}{black!30}{1}{\tiny} \\
%\textbf{French}  & C2 & %\pictofraction{\faCircle}{cvpurple}{3}{black!30}{1}{\tiny}
\end{tabular}
\end{minipage}
}



\bigskip
{
\Large	
\bg{cvgreen}{white}{Python Frameworks}\\
}
%\bubblediagram{{\textbf{a pirate's} \\\textbf{life}},  the sea, pillaging, plundering, ships, stealing, hijacking, \textbf{The Black}\\ \textbf{Pearl}}
{
\large	
\bubblediagram
{
	{\textbf{Python} \\\textbf{KI}}, Sklearn, Pandas, Numpy,	\textbf{TensorFlow}\\ \textbf{Pytorch}, Streamlit,
FastApi, {jupyter\\ notebook \\ VS code}}
}

{\Large
\hspace{1cm} \color{labelcolour}{OS:} \hspace{0.5em}\icon{\faWindows}{labelcolour}{\Large} \hspace{0.5em} \icon{\faLinux}{labelcolour}{\Large} 
}

\bigskip

{
\Large\bg{cvgreen}{white}{IT \& programming} \\
}
\begin{minipage}[t]{0.3\textwidth}
\begin{tabular}{r @{\hspace{0.5em}}l}
     \bg{skilllabelcolour}{iconcolour}{html, css} &  \barrule{0.4}{0.5em}{cvpurple}\\
     \bg{skilllabelcolour}{iconcolour}{\LaTeX} & \barrule{0.45}{0.5em}{cvpurple} \\
     %\bg{skilllabelcolour}{iconcolour}{xslt} & \barrule{0.5}{0.5em}{cvpurple} \\
     %\bg{skilllabelcolour}{iconcolour}{python} & \barrule{0.5}{0.5em}{cvpurple} \\
     %\bg{skilllabelcolour}{iconcolour}{R} & \barrule{0.25}{0.5em}{cvpurple} \\
     \bg{skilllabelcolour}{iconcolour}{javascript} & \barrule{0.1}{0.5em}{cvpurple} \\
     \bg{skilllabelcolour}{iconcolour}{docker, git, github} & \barrule{0.3}{0.5em}{cvpurple} \\
     \bg{skilllabelcolour}{iconcolour}{heroku} & \barrule{0.3}{0.5em}{cvpurple} \\
     \bg{skilllabelcolour}{iconcolour}{web-parsing} & \barrule{0.45}{0.5em}{cvpurple} \\
     
\end{tabular}

%\bgupper{cvgreen}{white}{Soft Skills} \\
%{\large
%	\cvtag{Fleissigkeit}
%	\cvtag{Ehrlichkeit}
%	\cvtag{Respektabilität}
%}


%\dashrule{}{}
\end{minipage}


\bigskip

\scalebox{0.8}{
\iconcross{\Huge}{white}{cvred}{\color{black!30}\faBook}{\href{mailto:the.latex.ninja@gmail.com}{\faEnvelopeO}}{\faPhone}{\faCode}
}

\phantom{turn the page}

\phantom{turn the page}
}
%-----------------------------------------------------------
\switchcolumn

\small
\section*{Berufliche Erfahrung}

\begin{tabular}{r| p{0.4\textwidth} c}
    \cvevent{2018--2025}{Eigener Projekt}{Analyse der Wahlen}{Deutschland \color{cvred}}{Application.}{Python} \\
    \cvevent{2022--2025}{Eigener Projekt}{Credit scoring}{Taiwan 2005 \color{cvred}}{Application.}{Python} \\
    %\cvevent{2019}{Freelance Pirate}{Bucaneering}{Tortuga \color{cvred}}{This and that. The usual, aye?}{medal.jpeg} \\
    %\cvevent{2016--2017}{Captain of the Black Pearl}{Lead}{Tortuga \color{cvred}}{Found a secret treasure, lost the ship.}{medal.jpeg}
\end{tabular}

\vspace{4em}

\begin{minipage}[t]{0.4\textwidth}
%\section*{Degrees}
\section*{Bildung}
\begin{tabular}{r p{0.6\textwidth} c}
    \cvdegree{2000-2004}{Bachelor}{Mathematik}{Uni \color{headerblue}}{}{uni_k} \\
    \cvdegree{2004-2006}{Master}{Mathematik}{Uni \color{headerblue}}{}{uni_k} \\
    %\cvdegree{1715}{Bucaneering}{M.A.}{London \color{headerblue}}{}{medal.jpeg} \\
    \cvdegree{2007-2010}{PhD}{Mathematik}{Uni \color{headerblue}}{}{uni_a.png}
\end{tabular}
\end{minipage}\hfill
\begin{minipage}[t]{0.16\textwidth}
\section*{Hobbies}
% usage \hobbyicon{<fontawesome icon}{Text}{background color of circle}{size of icon}{space text below icon}
\hobbyicon{\color{iconcolour}\faFlask}{Rhum}{cvgreen}{\iconsize}{2em} \hfill
\hobbyicon{\color{iconcolour}\faBook}{The Code}{cvorange}{\iconsize}{2em}

\hobbyicon{\color{iconcolour}\faComment}{Parler}{cvpurple}{\iconsize}{2em} \hspace{1em}
\hobbyicon{\color{iconcolour}\faBeer}{Beer}{headerblue}{\iconsize}{2em}
\end{minipage}

\vspace{4em}

\begin{minipage}[t]{0.3\textwidth}
\section*{Certificates}
\begin{tabular}{>{\footnotesize\bfseries}r >{\footnotesize}p{0.55\textwidth}}
	07/2022 - 09/2022 & Deutsch für den Beruf \\
	09/2022 - 09/2022 & Relationale Datenbanken mit SQL\\
    11/2022 - 12/2022 & Machine Learning \\
    01/2023 - 01/2023 & Deep Learning \\
    01/2023 - 02/2023 & Technisches Englisch 
    %1715--1716 & Grant from the Pirate's Company
\end{tabular}
%\section*{Strenghts}
%\section*{Soft Skills}
%\cvtag{honest}
%\cvtag{thieving}
%\cvtag{handsome}
%\section*{References}
\vspace{4em}
\section*{ Web-apps in github }
%\cvkeyword{Will Turner}{cvgreen}{iconcolour}
%\cvkeyword{Barbossa}{cvgreen}{iconcolour} \\

%\cvkeyword{possibly Mr. Swan}{headerblue}{iconcolour}
%\cvkeyword{ \href{https://github.com/AlexanderUhmann/streamlit_cs}{App Credit Scoring}}{headerblue}{iconcolour}\\
\href{https://github.com/AlexanderUhmann/streamlit_cs}{App Credit Scoring}

\end{minipage}\hfill
\begin{minipage}[t]{0.3\textwidth}
%\section*{Publications}
\section*{Projekte}
%\begin{tabular}{>{\footnotesize\bfseries}r >{\footnotesize}p{0.7\textwidth}}
%    1729 & \emph{How I almost got killed by Lady Swan}, Tortuga Printing Press. \\
%    1720 & ``Privateering for Beginners'', in: \emph{The Pragmatic Pirate} (1/1720).
%\end{tabular}

\begin{tabular}{r| p{0.4\textwidth} c}
%	\cvevent{2018--2025}{Eigener Projekt}{Analyse der Wahlen}{Deutschland \color{cvred}}{Application.}{Python} \\
%	\cvevent{2022--2025}{Eigener Projekt}{Credit scoring}{Taiwan 2005 \color{cvred}}{Application.}{Python} \\
	%\cvevent{2019}{Freelance Pirate}{Bucaneering}{Tortuga \color{cvred}}{This and that. The usual, aye?}{medal.jpeg} \\
	%\cvevent{2016--2017}{Captain of the Black Pearl}{Lead}{Tortuga \color{cvred}}{Found a secret treasure, lost the ship.}{medal.jpeg}
\end{tabular}

\begin{tabular}{>{\footnotesize\bfseries}r >{\footnotesize}p{0.7\textwidth}}
    2018--2025&\emph{Analyse der Wahlen} \\
    2022--2025&\emph{Credit scoring Taiwan 2005}
\end{tabular}


%\section*{Talks}
%\begin{tabular}{>{\footnotesize\bfseries}r >{\footnotesize}p{0.6\textwidth}}
%    Nov. 1726 & ``How I lost my ship (\& and how to get it back)'', at: \emph{Annual Pirate's Conference} in Tortuga, Nov. 1726.
%\end{tabular}

\end{minipage}









\vfill{} % Whitespace before final footer

%----------------------------------------------------------------------------------------
%	FINAL FOOTER
%----------------------------------------------------------------------------------------
\setlength{\parindent}{0pt}
%\begin{minipage}[t]{\rightcolwidth}
%\begin{center}\fontfamily{\sfdefault}\selectfont \color{black!70}
%{\small Alexander Uhmann \icon{\faEnvelopeO}{cvpurple}{} Sachsen \icon{\faMapMarker}{cvpurple}{} Leipzig \icon{\faPhone}{cvpurple}{} 0172/1854194 \newline\icon{\faAt}{cvpurple}{}	 
%{\Large \protect\url{  alexander.uhmann@gmail.com}\\ 
%\vspace{2em}
%\textit{\today}}	
%}
%\end{center}
%\end{minipage}


\end{paracol}

\end{document}
